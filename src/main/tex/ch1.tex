\documentclass[12pt, a4paper]{article}

\usepackage[T1]{fontenc}
\usepackage[a4paper, margin=1in]{geometry}
\usepackage{authblk}

\usepackage{amsmath}
\usepackage{bm}
\usepackage{amssymb}
\usepackage{amsfonts}

\usepackage{booktabs}
\usepackage{array}
\usepackage{graphicx}
\usepackage{float}

\usepackage{xcolor}
\usepackage{tcolorbox}

\usepackage[backend=biber, style=ieee, sorting=ynt]{biblatex}
\usepackage{url}
\addbibresource{citations.bib}
\addbibresource{ch1.bib}

\usepackage[parfill]{parskip}
\usepackage{enumitem}
\usepackage{hyperref}

%\graphicspath{{..\img} {.\img}}

%\newcommand\theequation{eq.\arabic{equation}}

\newcounter{txtrefcounter}
\makeatletter
\newcommand*{\labeltext}[2]{%
    \refstepcounter{txtrefcounter}%increment text-reference counter
    \edef\@currentlabel{phr.\,\arabic{txtrefcounter}}% Set target label
    \phantomsection% Correct hyper reference link
    \textcolor{txt-red}{#1}~[\textcolor{txt-blue}{phr.\,\arabic{txtrefcounter}}] \label{#2}% Print and store label
}
\makeatother

\definecolor{bg-black}{RGB}{42, 42, 42}
\definecolor{bg-white}{RGB}{255, 255, 255}
\definecolor{txt-blue}{RGB}{3, 169, 244}
\definecolor{txt-orange}{RGB}{255, 171, 0}
\definecolor{txt-yellow}{RGB}{255, 221, 75}
\definecolor{txt-red}{RGB}{229, 57 ,53}

\tcbset{
	width=\textwidth,
	colback=bg-white,
	colframe=bg-black,
	coltitle=txt-yellow,
	fonttitle=\sffamily,
	arc=0.1mm,
	left=1mm
}

\hypersetup{pdfinfo={
	Title={Chapter-1: Getting Started},  % unrelated to the title on the first-page; title of the morsel
	Subject={},  % unrelated to abstract or similar document content; abstract for the website
	Author={Harsha Vardhan}, % unrelated to the \author{} commands' content
	% CreationDate={}, % date format is D:YYYYMMDDHHmmSSOHH'mm' or D:YYYYMMDD
	% ModDate={}, % date format is D:YYYYMMDDHHmmSSOHH'mm' or D:YYYYMMDD
	CreatedOn={October 14, 2023}, % custom metadata-entry; format: Month Day, Year
	LastUpdatedOn={\today}, % custom metadata-entry
	Keywords={expressions | cpp} % separated by " | ", i.e. pipe symbol
}}

\begin{document}
    \title{Chapter 1: Getting Started~\cite{cpp-primer}}
    \author{Harsha Vardhan}
    \date{Last Updated: \today}

    \maketitle
    \bigskip

    \begin{tcolorbox}[title={Exercise: 1.1}]
        Review the documentation for your compiler and determine what file naming convention it uses.
        Compile and run the \texttt{main} program from page 2.
    \end{tcolorbox}

    \noindent \texttt{C++} source files conventionally use one of the suffixes \texttt{.C}, \texttt{.cc}, \texttt{.cpp}, \texttt{.CPP}, \texttt{.c++}, \texttt{.cp}, or \texttt{.cxx}; \texttt{C++} header files often use \texttt{.hh}, \texttt{.hpp}, \texttt{.H}, or (for shared template code) \texttt{.tcc}; and preprocessed \texttt{C++} files use the suffix \texttt{.ii}.
    GCC recognizes files with these names and compiles them as \texttt{C++} programs \dots \cite{gcc-manpage}.

    \noindent The following shell-script snippet, run from the \texttt{src\textbackslash main\textbackslash cpp\textbackslash ch1} directory, will compile and run the requisite source-file.
    \begin{verbatim}
        $> g++ e1.1.cpp -o e1.1 -std=c++0x
        $> ./e1.1
        $> echo "Exit code: ${?}"
        Exit code: 0
    \end{verbatim}

    \noindent\textbf{Note:} one may notice that the output file-name doesn't include the file-extension (i.e., the above snippet will work on both Windows PC and POSIX system).
    The \texttt{.exe} file-extension is required when one wants to launch the compiled application from Windows GUI (usually through a double click); executing from the terminal, by calling \texttt{./e1.1}, usually works even without the \texttt{.exe} file-extension.

    \bigskip
    \begin{tcolorbox}[title={Exercise: 1.2}]
        Change the program to return $-1$.
        A return value of $-1$ is often treated as an indicator that the program failed.
        Recompile and rerun your program to see how your system treats a failure indicator from main.
    \end{tcolorbox}

    \noindent The following shell-script snippet, run from the \texttt{src\textbackslash main\textbackslash cpp\textbackslash ch1} directory, will compile and run the requisite source-file.
    \begin{verbatim}
        $> g++ e1.2.cpp -o e1.2 -std=c++0x
        $> ./e1.1
        $> echo "Exit code: ${?}"
        Exit code: 127
    \end{verbatim}

    \noindent\textbf{Note:} The exit code of \texttt{127}

    \bigskip
    \begin{tcolorbox}[title={Exercise: 1.2}]
        Change the program to return $-1$.
        A return value of $-1$ is often treated as an indicator that the program failed.
        Recompile and rerun your program to see how your system treats a failure indicator from main.
    \end{tcolorbox}

    \pagebreak
    \printbibliography

\end{document}
