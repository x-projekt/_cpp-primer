\documentclass[12pt, a4paper]{article}

\usepackage[T1]{fontenc}
\usepackage[a4paper, margin=1in]{geometry}
\usepackage{authblk}

\usepackage{amsmath}
\usepackage{bm}
\usepackage{amssymb}
\usepackage{amsfonts}

\usepackage{booktabs}
\usepackage{array}
\usepackage{graphicx}
\usepackage{float}

\usepackage{xcolor}
\usepackage{tcolorbox}

\usepackage[backend=biber, style=ieee, sorting=ynt]{biblatex}
\usepackage{url}
\addbibresource{citations.bib}
\addbibresource{ch1.bib}

\usepackage[parfill]{parskip}
\usepackage{enumitem}
\usepackage{hyperref}

%\graphicspath{{..\img} {.\img}}

%\newcommand\theequation{eq.\arabic{equation}}

\newcounter{txtrefcounter}
\makeatletter
\newcommand*{\labeltext}[2]{%
    \refstepcounter{txtrefcounter}%increment text-reference counter
    \edef\@currentlabel{phr.\,\arabic{txtrefcounter}}% Set target label
    \phantomsection% Correct hyper reference link
    \textcolor{txt-red}{#1}~[\textcolor{txt-blue}{phr.\,\arabic{txtrefcounter}}] \label{#2}% Print and store label
}
\makeatother

\definecolor{bg-black}{RGB}{42, 42, 42}
\definecolor{bg-white}{RGB}{255, 255, 255}
\definecolor{txt-blue}{RGB}{3, 169, 244}
\definecolor{txt-orange}{RGB}{255, 171, 0}
\definecolor{txt-yellow}{RGB}{255, 221, 75}
\definecolor{txt-red}{RGB}{229, 57 ,53}

\tcbset{
	width=\textwidth,
	colback=bg-white,
	colframe=bg-black,
	coltitle=txt-yellow,
	fonttitle=\sffamily,
	arc=0.1mm,
	left=1mm
}

\hypersetup{pdfinfo={
	Title={Chapter-1: Getting Started},  % unrelated to the title on the first-page; title of the morsel
	Subject={},  % unrelated to abstract or similar document content; abstract for the website
	Author={Harsha Vardhan}, % unrelated to the \author{} commands' content
	% CreationDate={}, % date format is D:YYYYMMDDHHmmSSOHH'mm' or D:YYYYMMDD
	% ModDate={}, % date format is D:YYYYMMDDHHmmSSOHH'mm' or D:YYYYMMDD
	CreatedOn={October 14, 2023}, % custom metadata-entry; format: Month Day, Year
	LastUpdatedOn={\today}, % custom metadata-entry
	Keywords={expressions | cpp} % separated by " | ", i.e. pipe symbol
}}

\begin{document}
    \title{Chapter 1: Getting Started~\cite{cpp-primer}}
    \author{Harsha Vardhan}
    \date{Last Updated: \today}

    \maketitle
    \bigskip

    \noindent\textbf{Note:} all shell-script snippets in this document must be run from \texttt{src\textbackslash main\textbackslash cpp\textbackslash ch1} directory, unless stated otherwise.
    \bigskip

    \begin{tcolorbox}[title={Exercise: 1.1}]
        Review the documentation for your compiler and determine what file naming convention it uses.
        Compile and run the \texttt{main} program from page 2.
    \end{tcolorbox}
    \noindent\textbf{Source-file(s)}: \texttt{\textbackslash src\textbackslash main\textbackslash cpp\textbackslash e1.1.cpp}

    \noindent \texttt{C++} source files conventionally use one of the suffixes \texttt{.C}, \texttt{.cc}, \texttt{.cpp}, \texttt{.CPP}, \texttt{.c++}, \texttt{.cp}, or \texttt{.cxx}; \texttt{C++} header files often use \texttt{.hh}, \texttt{.hpp}, \texttt{.H}, or (for shared template code) \texttt{.tcc}; and preprocessed \texttt{C++} files use the suffix \texttt{.ii}.
    GCC recognizes files with these names and compiles them as \texttt{C++} programs \dots \cite{gcc-manpage}.

    \noindent The following shell-script snippet will compile and run the requisite source-file.
    \begin{verbatim}
        $> g++ e1.1.cpp -o e1.1 -std=c++0x
        $> ./e1.1
        $> echo "Exit code: ${?}"
        Exit code: 0
    \end{verbatim}

    \noindent\textbf{Note:} one may notice that the output file-name doesn't include the file-extension (i.e., the above snippet will work on both Windows PC and POSIX system).
    The \texttt{.exe} file-extension is required when one wants to launch the compiled application from Windows GUI (usually through a double click); executing from the terminal, by calling \texttt{./e1.1}, usually works even without the \texttt{.exe} file-extension.

    \bigskip
    \begin{tcolorbox}[title={Exercise: 1.2}]
        Change the program to return $-1$.
        A return value of $-1$ is often treated as an indicator that the program failed.
        Recompile and rerun your program to see how your system treats a failure indicator from main.
    \end{tcolorbox}
    \noindent\textbf{Source-file(s)}: \texttt{\textbackslash src\textbackslash main\textbackslash cpp\textbackslash e1.2.cpp}

    \noindent The following shell-script snippet will compile and run the requisite source-file.
    \begin{verbatim}
        $> g++ e1.2.cpp -o e1.2 -std=c++0x
        $> ./e1.1
        $> echo "Exit code: ${?}"
        Exit code: 127
    \end{verbatim}

    \noindent\textbf{Note:} exit-codes can only take on values from \texttt{0} to \texttt{255} (both inclusive).
    All values outside this range will simpy wrap around; exit code is computed as $(n - \lfloor\frac{n}{256}\rfloor * 256)\,\,\forall n > 0$.
    However, (on \texttt{GNU bash, version 5.2.15(1)-release (x86\_64-pc-msys)}) all negative exit codes (regardless of their absolute value) are mapped to \texttt{127}.
    This behaviour depends on the shell we use.

    \noindent For example, on RedHat Linux: If the exit code is a negative number, the resulting exit status is that number subtracted from 256.
    So, if your exit code is 20, then the exit status is 236~\cite{rhl-neg-exit-code}.

    \bigskip
    \begin{tcolorbox}[title={Exercise: 1.3}]
        Write a program to print \texttt{Hello, World} on the standard output.
    \end{tcolorbox}
    \noindent\textbf{Source-file(s)}: \texttt{\textbackslash src\textbackslash main\textbackslash cpp\textbackslash e1.3.cpp}

    \noindent The following shell-script snippet will compile and run the requisite source-file.
    \begin{verbatim}
        $> g++ e1.3.cpp -o e1.3 -std=c++0x
        $> ./e1.3
        Hello, World
    \end{verbatim}

    \bigskip
    \begin{tcolorbox}[title={Exercise: 1.4}]
        Our program used the addition operator, $+$, to add two numbers.
        Write a program that uses the multiplication, $*$, to print the product instead.
    \end{tcolorbox}
    \noindent\textbf{Source-file(s)}: \texttt{\textbackslash src\textbackslash main\textbackslash cpp\textbackslash e1.4.cpp}

    \noindent The following shell-script snippet will compile and run the requisite source-file.
    \begin{verbatim}
        $> g++ e1.4.cpp -o e1.4 -std=c++0x
        $> ./e1.4
        Enter two numbers:
        25.4 32.9
        Product of 25.4 and 32.9 is 835.66
    \end{verbatim}

    \bigskip
    \begin{tcolorbox}[title={Exercise: 1.5}]
        We wrote the output in one large statement.
        Rewrite the program to use a separate statement to print each operand.
    \end{tcolorbox}
    \noindent\textbf{Source-file(s)}: \texttt{\textbackslash src\textbackslash main\textbackslash cpp\textbackslash e1.5.cpp}

    \noindent The following shell-script snippet will compile and run the requisite source-file.
    \begin{verbatim}
        $> g++ e1.5.cpp -o e1.5 -std=c++0x
        $> ./e1.5
        Enter two numbers:
        25.4 32.9
        Product of 25.4 and 32.9 is 835.66
    \end{verbatim}

    \bigskip
    \begin{tcolorbox}[title={Exercise: 1.6}]
        Explain whether the following program fragment is legal.

        \begin{verbatim}
            std::cout << "The sum of " << v1;
                      << " and " << v2;
                      << " is " << v1 + v2 << std::endl;
        \end{verbatim}

        If the program is legal, what does it do?
        If the program is not legal, why not?
        How would you fix it?
    \end{tcolorbox}
    \noindent\textbf{Source-file(s)}: \texttt{\textbackslash src\textbackslash main\textbackslash cpp\textbackslash e1.6.cpp}

    \noindent The code snippet is illegal, and will result in a syntax error during compilation;
    The line-termination symbol (i.e. \texttt{;}) at the end of each line makes each a statement in itself.
    Further,
    \begin{itemize}
        \item {The first-line is syntactically legal. However, since we are not explicitly flushing the buffer, it's possible some of the output is not printed onto the standard output.}
        \item {The second and third lines are syntactically illegal; the output-operator (i.e. \texttt{<<}) has no left-hand operand, which must be an output-stream object.}
    \end{itemize}

    \noindent There are many ways of fixing the above snippet, two of which are mentioned in the requisite source-file.

    \bigskip
    \begin{tcolorbox}[title={Exercise: 1.7}]
        Compile a program that has incorrectly nested comments.
    \end{tcolorbox}
    \noindent\textbf{Source-file(s)}: \texttt{\textbackslash src\textbackslash main\textbackslash cpp\textbackslash e1.8.cpp}

    \noindent The following shell-script snippet will compile and run the requisite source-file.
    \begin{verbatim}
        $> g++ e1.8.cpp -o e1.8 -std=c++0x
        $> ./e1.8
        e1.8.cpp:9:24: warning: missing terminating " character
            9 |     std::cout << /* "*/" */;
              |                        ^
        e1.8.cpp:9:24: error: missing terminating " character
            9 |     std::cout << /* "*/" */;
              |                        ^~~~~
        e1.8.cpp: In function 'int main()':
        e1.8.cpp:9:15: error: no match for 'operator<<' (operand types are 'std::ostream' {aka 'std::basic_ostream<char>'} and 'std::ostream' {aka 'std::basic_ostream<char>'})
            9 |     std::cout << /* "*/" */;
              |     ~~~~~~~~~ ^~
              |          |
              |          basic_ostream<[...]>
           10 |
           11 |     std::cout << /* "*/" /* "/*" */; // legal
              |     ~~~~~~~~~
              |          |
              |          basic_ostream<[...]>
        e1.8.cpp:9:15: note: candidate: 'operator<<(int, int)' (built-in)
            9 |     std::cout << /* "*/" */;
              |     ~~~~~~~~~~^~~~~~~~~~~~~~
           10 |
              |
           11 |     std::cout << /* "*/" /* "/*" */; // legal
              |     ~~~~~~~~~
        e1.8.cpp:9:15: note:   no known conversion for argument 2 from 'std::ostream' {aka 'std::basic_ostream<char>'} to 'int'
        ...
    \end{verbatim}

    \bigskip
    \begin{tcolorbox}[title={Exercise: 1.8}]
        Indicate which, if any, of the following output statements are legal:
        \begin{verbatim}
            std::cout << "/*";
            std::cout << "*/";
            std::cout << /* "*/" */;
            std::cout << /* "*/" /* "/*" */;
        \end{verbatim}
        After you've predicted what will happen, test your answers by compiling a program with each of these statements.
        Correct any errors you encounter.
    \end{tcolorbox}
    \noindent\textbf{Source-file(s)}: \texttt{\textbackslash src\textbackslash main\textbackslash cpp\textbackslash e1.8.cpp}

    \noindent Here, first, second, and fourth lines are syntactically accurate.
    However, third line is in-accurate and can be fixed by inserting double-quotes between the lines' last \texttt{/} and \texttt{;} characters, i.e.
    \begin{verbatim}
        std::cout << /* "*/" */";
    \end{verbatim}

    \noindent\textbf{Note:} see solution for \textit{Exercise: 1.7} for output of compiling the erroneous snippet.

    \bigskip
    \begin{tcolorbox}[title={Exercise: 1.9}]
        Write a program that uses a \texttt{while} to sum the numbers from $50$ to $100$.
    \end{tcolorbox}
    \noindent\textbf{Source-file(s)}: \texttt{\textbackslash src\textbackslash main\textbackslash cpp\textbackslash e1.9.cpp}

    \noindent The following shell-script snippet will compile and run the requisite source-file.
    \begin{verbatim}
        $> g++ ./e1.9.cpp -o e1.9 -std=c++0x
        $> ./e1.9
        Sum from 50 to 100 (integers): 3825
    \end{verbatim}

    \bigskip
    \begin{tcolorbox}[title={Exercise: 1.10}]
        In addition to the $++$ operator that adds $1$ to its operands, there is a decrement operator ($--$) that subtracts $1$.
        Use the decrement operator to write a \texttt{while} that prints the numbers from ten down to zero.
    \end{tcolorbox}
    \noindent\textbf{Source-file(s)}: \texttt{\textbackslash src\textbackslash main\textbackslash cpp\textbackslash e1.10.cpp}

    \noindent The following shell-script snippet will compile and run the requisite source-file.
    \begin{verbatim}
        $> g++ ./e1.10.cpp -o e1.10 -std=c++0x
        $> ./e1.10
        10
        9
        8
        7
        6
        5
        4
        3
        2
        1
        0
    \end{verbatim}

    \bigskip
    \begin{tcolorbox}[title={Exercise: 1.11}]
        Write a program that prompts the user for two integers.
        Print each number in the range specified by those two integers.
    \end{tcolorbox}
    \noindent\textbf{Source-file(s)}: \texttt{\textbackslash src\textbackslash main\textbackslash cpp\textbackslash e1.11.cpp}

    \noindent The following shell-script snippet will compile and run the requisite source-file.
    \begin{verbatim}
        $> g++ ./e1.11.cpp -o e1.11 -std=c++0x
        $> ./e1.11
        Enter start number: 6
        Enter end number: 2
        5
        4
        3
        $> ./e1.11
        Enter start number: 10
        Enter end number: 13
        11
        12
    \end{verbatim}

    \bigskip
    \begin{tcolorbox}[title={Exercise: 1.12}]
        What does the following \texttt{for} loop do?
        What is the final value of \texttt{sum}?
        \begin{verbatim}
            int sum = 0;
            for (int i = -100; i <= 100; ++i)
                sum += i;
        \end{verbatim}
    \end{tcolorbox}
    \noindent\textbf{Source-file(s)}: \texttt{\textbackslash src\textbackslash main\textbackslash cpp\textbackslash e1.12.cpp}

    \noindent The above snippet sums integers from $-100$ to $100$, inclusive of both; resulting in a total sum of $0$.
    The following shell-script snippet will compile and run the requisite source-file.
    \begin{verbatim}
        $> g++ ./e1.12.cpp -o e1.12 -std=c++0x
        $> ./e1.12
        Sum: 0
    \end{verbatim}

    \bigskip
    \begin{tcolorbox}[title={Exercise: 1.13}]
        Rewrite the exercises from \texttt{1.4.1} (p.~13) using \texttt{for} loops.
    \end{tcolorbox}
    \noindent\textbf{Source-file(s)}:
    \\ \texttt{\textbackslash src\textbackslash main\textbackslash cpp\textbackslash e1.13-1.9.cpp}
    \\ \texttt{\textbackslash src\textbackslash main\textbackslash cpp\textbackslash e1.13-1.10.cpp}
    \\ \texttt{\textbackslash src\textbackslash main\textbackslash cpp\textbackslash e1.13-1.11.cpp}

    \noindent The following shell-script snippet will compile and run the requisite source-file(s).
    \begin{verbatim}
        $> g++ ./e1.13-1.9.cpp -o e1.13-1.9 -std=c++0x
        $> ./e1.13-1.9
        Sum of integers from 50 to 100: 3825
    \end{verbatim}

    \begin{verbatim}
        $> g++ ./e1.13-1.10.cpp -o e1.13-1.10 -std=c++0x
        $> ./e1.13-1.10
        10
        9
        8
        7
        6
        5
        4
        3
        2
        1
        0
    \end{verbatim}

    \begin{verbatim}
        $> g++ ./e1.13-1.11.cpp -o e1.13-1.11 -std=c++0x
        $> ./e1.13-1.11
        Enter the first number: 29
        Enter the second number: 34
        30
        31
        32
        33
        $> ./e1.13-1.11
        Enter the first number: 34
        Enter the second number: 29
        33
        32
        31
        30
    \end{verbatim}

    \bigskip
    \begin{tcolorbox}[title={Exercise: 1.14}]
        Compare and contrast the loops that used a \texttt{for} with those using a \texttt{while}
        Are there advantages or disadvantages to using either form?
    \end{tcolorbox}

    \noindent The internet is littered with comparisons between \texttt{for} and \texttt{while} loops; and are usually language agnostic.
    Read the StackOverflow post titled: \texttt{For vs. while in C programming?}~\cite{stackoverflow-for-while}.

    \bigskip
    \begin{tcolorbox}[title={Exercise: 1.15}]
        Write programs that contain the common errors discussed in the box on page 16.
        Familiarize yourself with the message the compiler generates.
    \end{tcolorbox}
    \noindent\textbf{Source-file(s):} \texttt{\textbackslash src\textbackslash main\textbackslash cpp\textbackslash e1.15.cpp}

    \noindent\textbf{Note:} uncomment a single erroneous line (in the above source-file) before compiling; to see the compilation error corresponding to that line.

    \bigskip
    \begin{tcolorbox}[title={Exercise: 1.16}]
        Write your own version of a program that prints the sum of a set of integers read from \texttt{cin}.
    \end{tcolorbox}
    \noindent\textbf{Source-file(s):} \texttt{\textbackslash src\textbackslash main\textbackslash cpp\textbackslash e1.16.cpp}

    \noindent The following shell-script snippet will compile and run the requisite source-file(s).
    \begin{verbatim}
        $> g++ ./e1.16.cpp -o e1.16 -std=c++0x
        $> ./e1.16
        1
        2
        3
        4
        Sum: 10
    \end{verbatim}

    \bigskip
    \begin{tcolorbox}[title={Exercise: 1.17}]
        What happens in the program presented in this section if the input values are all equal?
        What if there are no duplicated values?
    \end{tcolorbox}
    \noindent It doesn't matter if all the inputs are equal or different.
    If all inputs are equal, the program counts the number of times the input has occurred.
    Whereas, if all the inputs are different, the program outputs a count of $1$ for each input.

    \noindent\textbf{Note:} See the solution for \textit{Exercise-1.18} for proof of the above solution.

    \bigskip
    \begin{tcolorbox}[title={Exercise: 1.18}]
        Compile and run the program from this section giving it only equal values as input.
        Run it again giving it values in which no number is repeated.
    \end{tcolorbox}
    \noindent\textbf{Source-file(s):} \texttt{\textbackslash src\textbackslash main\textbackslash cpp\textbackslash e1.18.cpp}

    \noindent The following shell-script snippet will compile and run the requisite source-file(s).
    \begin{verbatim}
        $> g++ ./e1.18.cpp -o e1.18 -std=c++0x
        $> ./e1.18
        1
        1
        1
        ^D
        1 occurs 3 times
        $> ./e1.18
        1
        2
        1 occurs 1 times
        3
        2 occurs 1 times
        ^D
        3 occurs 1 times
    \end{verbatim}

    \bigskip
    \begin{tcolorbox}[title={Exercise: 1.19}]
        Revise the program you wrote for the exercises in \texttt{1.4.1} (p.~13) that printed a range of numbers so that it handles input in which the first number is smaller than second.
    \end{tcolorbox}
    \noindent\textbf{Source-file(s):}
    \\ \texttt{\textbackslash src\textbackslash main\textbackslash cpp\textbackslash e1.11.cpp}
    \\ \texttt{\textbackslash src\textbackslash main\textbackslash cpp\textbackslash e1.13-1.11.cpp}

    \noindent\textbf{Note:} See the solution(s) for \textit{Exercise-1.11} and \textit{Exercise-1.13}

    \bigskip
    \begin{tcolorbox}[title={Exercise: 1.20}]
        \texttt{http://www.informit.com/title/032174113} contains a copy of \texttt{Sales\_item.h} in the Chapter 1 code directory.
        Copy that file to your working directory.
        Use it to write a program that reads a set of books sales transactions, writing each transaction to the standard output.
    \end{tcolorbox}
    \noindent\textbf{Source-file(s):}
    \\ \texttt{\textbackslash src\textbackslash main\textbackslash cpp\textbackslash e1.20.cpp}
    \\ \texttt{\textbackslash src\textbackslash main\textbackslash cpp\textbackslash Sales\_item.h}

    \bigskip
    \begin{tcolorbox}[title={Exercise: 1.21}]
        Write a program that reads two \texttt{Sales\_item} objects that hve the same ISBN and produces their sum.
    \end{tcolorbox}
    \noindent\textbf{Source-file(s):}
    \\ \texttt{\textbackslash src\textbackslash main\textbackslash cpp\textbackslash e1.21.cpp}
    \\ \texttt{\textbackslash src\textbackslash main\textbackslash cpp\textbackslash Sales\_item.h}

    \noindent The following shell-script snippet will compile and run the requisite source-file(s).
    \begin{verbatim}
        $> g++ ./e1.21.cpp -o e1.21 -std=c++0x
        $> ./e1.21
        Enter first item: 0-201-78345-X 3 26.54
        Enter second item: 0-201-78345-X 13 35.06
        Items sum: 0-201-78345-X 16 535.4 33.4625
        $> ./e1.21
        Enter first item: 0-201-78345-X 3 26.54
        Enter second item: 0-201-78345-Y 12 78.54
        Invalid Input: items have different ISBN
    \end{verbatim}

    \bigskip
    \begin{tcolorbox}[title={Exercise: 1.22}]
        Write a program that reads several transactions for the same ISBN .
        Write the sum of all the transactions that were used.
    \end{tcolorbox}
    \noindent\textbf{Source-file(s):}
    \\ \texttt{\textbackslash src\textbackslash main\textbackslash cpp\textbackslash e1.22.cpp}
    \\ \texttt{\textbackslash src\textbackslash main\textbackslash cpp\textbackslash Sales\_item.h}

    \noindent The following shell-script snippet will compile and run the requisite source-file(s).
    \begin{verbatim}
        $> g++ ./e1.22.cpp -o e1.22 -std=c++0x
        $> ./e1.22
        Enter item: 0-201-78345-X 3 26.54
        Enter item: 0-201-78345-X 9 32.08
        Enter item: 0-201-78345-Y 10 322.00
        Expected same ISBN [0-201-78345-X], but got [0-201-78345-Y]; Discarding last transaction.
        Discarded transaction: 0-201-78345-Y 10 3220 322
        Enter item: 0-201-78345-X 15 30.39
        Enter item: ^D
        ^D
        Sum (until now): 0-201-78345-X 27 824.19 30.5256
    \end{verbatim}

    \bigskip
    \begin{tcolorbox}[title={Exercise: 1.23}]
        Write a program that reads several transactions and counts how many transactions occurs for each ISBN.
    \end{tcolorbox}
    \noindent\textbf{Source-file(s):}
    \\ \texttt{\textbackslash src\textbackslash main\textbackslash cpp\textbackslash e1.23.cpp}
    \\ \texttt{\textbackslash src\textbackslash main\textbackslash cpp\textbackslash Sales\_item.h}

    \noindent The following shell-script snippet will compile and run the requisite source-file(s).
    \begin{verbatim}
        $> g++ ./e1.23.cpp -o e1.23 -std=c++0x
        $> ./e1.23
        Enter item: 0-201-78345-X 3 26.54
        Enter item: 0-201-78345-X 2 25.94
        Enter item: 0-201-78345-X 5 29.62
        Enter item: 0-201-78345-Y 3 66.54
        ISBN [0-201-78345-X] occurs #3 times
        Enter item: 0-201-78345-Y 9 69.00
        Enter item: ^D
        ^D
        ISBN [0-201-78345-Y] occurs #2 times
    \end{verbatim}

    \bigskip
    \begin{tcolorbox}[title={Exercise: 1.24}]
        Test the previous program by giving multiple transactions representing multiple ISBNs.
        The records for each ISBN should be grouped together.
    \end{tcolorbox}
    \noindent\textbf{Source-file(s):}
    \\ \texttt{\textbackslash src\textbackslash main\textbackslash cpp\textbackslash e1.23.cpp}
    \\ \texttt{\textbackslash src\textbackslash main\textbackslash cpp\textbackslash Sales\_item.h}

    \noindent The program written for \textit{Exercise-1.23} only groups (and thereby counts) transactions occurring consecutively.
    For a fully functional program, one would require a data-structure that can hold key-value pairs (such as hash-map or dictionary).

    \noindent\textbf{Note:} see the solution for \textit{Exercise-1.23} for instructions about running the source-file(s).

    \bigskip
    \begin{tcolorbox}[title={Exercise: 1.25}]
        Using the \texttt{Sales\_item.h} header from the Web site, compile and execute the bookstore program presented in this section.
    \end{tcolorbox}
    \noindent\textbf{Source-file(s):}
    \\ \texttt{\textbackslash src\textbackslash main\textbackslash cpp\textbackslash e1.25.cpp}
    \\ \texttt{\textbackslash src\textbackslash main\textbackslash cpp\textbackslash Sales\_item.h}

    \noindent The following shell-script snippet will compile and run the requisite source-file(s).
    \begin{verbatim}
        $> g++ e1.25.cpp -o e1.25 -std=c++0x
        $> ./e1.25
        Enter item: 0-201-78345-X 3 26.54
        Enter item: 0-201-78345-X 2 25.94
        Enter item: 0-201-78345-X 5 29.62
        Enter item: 0-201-78345-Y 3 66.54
        Sum: 0-201-78345-X 10 279.6 27.96
        Enter item: 0-201-78345-Y 9 69.00
        Enter item: ^D
        ^D
        Sum: 0-201-78345-Y 12 820.62 68.385
    \end{verbatim}

    \pagebreak
    \printbibliography

\end{document}
